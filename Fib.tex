\documentclass[UTF8]{ctexart}
\title{斐波那契数列实验报告}
\author{万宗祺 201600090059}
\date{\today}
\usepackage{amsthm}
\usepackage{geometry}
\usepackage{amsfonts}
\usepackage{mathrsfs}
\usepackage{amsmath}
\usepackage{listings}
\newtheorem*{note}{注}
\geometry{left = 3.5cm,right=3.5cm,top=2.5cm,bottom=2.5cm}
\newcommand{\nb}{^\circ}
\begin{document}
\maketitle
\tableofcontents
\section{问题描述}
Fibonacci数列可由式$a_k = a_{k-1}+a_{k-2},k=3,4,\cdots$生成,其中初值$a_1=a_2=1$,要编写函数计算该数列的第n项,要求:

1.函数执行时首先测试输入参数,确保函数能正确调用;

2.分别使用递归和循环的方式实现,如$y=fibr(n)$递归方式,$y=fibl(n)$循环方式;

3.针对不同的n值,分别调用两种函数进行计算,并使用tic和toc函数计时,比较两种方式的计算时间,并分析原因。

\section{实验过程}
下面给出fibr.m与fibl.m的代码,内含详细的注释
\begin{note}
由于listing宏包不支持中文注释,注释只能写英文了
\end{note}
\lstset{language=Matlab}
\begin{lstlisting}
% fibr.m
function fib = fibr(n)
% Calculate item n recursively 
if n==1 || n==2 
% Boundary conditions at the end of recursion 
    fib = 1;
else
    fib = fibr(n-1)+fibr(n-2); 
% If do not get to the boundary, continue to call the function
end
\end{lstlisting}

\begin{lstlisting}
% fibl.m
function fib = fibl(n)
% Calculate item n circularly
fibb = [1,1];% set initial value
for i=3:n 
    fibb = [fibb fibb(i-1)+fibb(i-2)]; % calculate a new item
end
fib = fibb(n);
\end{lstlisting}

接下来,针对不同的n,分别调用两种函数进行计算并计时,脚本代码如下:

\begin{lstlisting}
%test.m
for n=21:25
    tic
    fibr(n);
    toc
    tic
    fibl(n);
    toc
end
\end{lstlisting}
\section{结果分析}
测试了计算n从21到25,两个函数分别需要的时间,结果以表格展示如下
\\
\\
\centerline{
\begin{tabular}{cccccc}
\hline
n & 21 & 22 &23 & 24 &25 \\
\hline
递归方式用时(s) & 0.107574 & 0.160348 & 0.240618 & 0.389781 &0.633730 \\
\hline
循环方式用时(s) & 0.000206&0.000067&0.000046&0.000071&0.000073 \\
\hline
\end{tabular}
}\\
\\
由此可以看出,循环方式所需要的时间远远少于递归方式,分析其原因,应该是递归方式进行了大量的重复计算,而循环方式顺推到n,对序列中每一个项只计算了一遍,而递归方式却是随着n增大,计算次数以指数方式增长。
\\
\end{document}