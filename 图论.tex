\documentclass[UTF8]{ctexart}
\title{图论讲义}
\author{万宗祺}
\date{\today}
\usepackage{amsthm}
\usepackage{geometry}
\usepackage{amsfonts}
\usepackage{mathrsfs}
\usepackage{amsmath}
\geometry{left = 2cm,right=2cm,top=2.5cm,bottom=2.5cm}
\newtheorem{dfnt}{定义}
\newtheorem{thr}{定理}
\newtheorem{lemma}{引理}
\newtheorem*{coro}{推论}
\newtheorem*{note}{注}
\newtheorem{pro}{命题}

\begin{document}
\maketitle
\tableofcontents
\section{图的匹配,覆盖和填装}
这一节介绍了图论中的重要问题--匹配(matching),覆盖(covering)和填装(packaging)

\textbf{独立}边构成的集合M称为一个匹配,若U中的每个顶点都与M中一个边关联,就称M是U的一个匹配,或者说U中顶点被M匹配,而不与M中顶点相关联的就称为非匹配顶点

给定图族$\mathscr{H}$,匹配问题的一个推广就是在已知图G中寻找尽可能多的顶点不相交的子图,使每个子图与$\mathscr{H}$中一个图同构,这就是填装问题。

而覆盖问题是指在图G中找到一个尽可能小的顶点集,使得G中任意一个与$\mathscr{H}$中图同构的子图都与该顶点集相交。

\subsection{二部图中的匹配}
约定该节中G为二部图,A,B是它的两个部分
\begin{dfnt}
\textbf{交错路}(alternating path)是指G中一条从A中非匹配点出发,其边在E$\backslash$ M和M中交错出现的路,若交错路终止于一个非匹配顶点,则称其为一条\textbf{增广路}(augmenting path)。
\end{dfnt}

通过增广路和M的对称差可以构造一个更大的匹配,不断进行直到没有增广路后,达到的匹配就是最优的匹配(拥有最大边数)
 
\begin{dfnt}[Konig]
对于$U \subset V$,如果G中每一条边都与U中一个顶点关联,就称U是E的一个\textbf{顶点覆盖}
\end{dfnt}
\begin{thr}[Hall]
二部图G中匹配的最大基数等于其边的顶点覆盖的最小基数。
\end{thr}


要寻找1-因子,直接的想法是找到一个饱和A匹配,如果还有$|A|=|B|$那么,自然这个匹配就是一个1-因子了

下面的定理说明了饱和A匹配在什么情况下存在
\begin{thr}[婚姻定理(marriage theorem)]
二部图G包含饱和A的匹配,当且仅当对所有的$S \subseteq A$,均有$|N(S)| \geq |S|$
\end{thr}
\begin{coro}
若G是k-正则的,则G包含1-因子(1-正则的支撑子图)
\end{coro}
\begin{coro}
2k-正则图都有2-因子
\end{coro}
\begin{note}
用了将每个点拆分为两个点从而构造二部图的技巧
\end{note}
\begin{dfnt}
由$E(V)$上的线性序$\leq_v$所形成的序族$(\leq_v)_{v \in V}$被称为G的一个\textbf{优先集},如果对每条边$e \in E \backslash M$,都存在一条边$f \in M$使得e和f有一个公共顶点v,且$e <_v v$,则称匹配M在G中是\textbf{稳定的}
\end{dfnt}
\begin{thr}[稳定婚姻定理(stable marriage theorem)]
对任意给定优先集,二部图G都有一个稳定匹配。
\end{thr}
\subsection{一般图中的匹配}
\begin{dfnt}
给定图G,我们用$C_G$表示它的分支的集合,用$q(G)$表示$C_G$中奇分支(阶为奇数)的个数
\end{dfnt}
\begin{thr}
图G有1-因子当且仅当对所有$S \subseteq V(G)$,有$q(G-S) \leq |S|$
\end{thr}
\begin{coro}
每个无桥的立方图都有1-因子
\end{coro}
\begin{dfnt}
非空图G=(V,E)称为\textbf{因子临界}的,如果G不包含1-因子,但对每个顶点v,G-v都包含1-因子。

令$G_S$是由G通过把分支$C \in C_{G-S}$收缩成单个顶点,并删除S内部的点而得到的二部图

如果$G_S$包含饱和S的一个匹配,就称S与$C_{G-S}$是\textbf{可匹配的}
\end{dfnt}
\begin{thr}[Gallai-Edmonds 匹配定理]
每个图G=(V,E)都包含一个顶点集S满足下面两条性质:\\(i)S与$C_{G-S}$是可匹配的;\\(ii)G-S的每个分支都是因子临界的。\\给定任意一个这样的S,G包含1-因子当且仅当$|S|=|C_{G-S}|$.
\end{thr}
\subsection{填装和覆盖}
\begin{dfnt}
在图族$\mathscr{H}$(可能相同的)到G的填装问题中,最多可能填装的图个数为k,将k和G中可以覆盖$\mathscr{H}$的所有子图的顶点最小数s进行比较,如果s有一个独立于G的,关于k的函数作为上界,则称$\mathscr{H}$具有Erdos-Posa性质。
\end{dfnt}
\begin{lemma}
$k \in \mathbb{N}$,且H是一个立方多重图。若$|H| \geq s_k$,则H包含k个不相交的圈。其中$$s_k=\left \{
\begin{aligned}
4kr_k &\quad& k \geq 2 \\
1 &\quad&k \leq 1
\end{aligned}
\right.
\quad \quad \quad r_k = \log k+\log \log k +4
$$
\end{lemma}
\begin{thr}[Erdos-Posa]
存在一个函数$f:\mathbb{N}\rightarrow \mathbb{N}$,使得对任意$k \in \mathbb{N}$,每个图或者包含k个不相交的圈,或者存在一个至多有f(k)个顶点得到集合与所有圈相交。
\end{thr}
\subsection{树填装}
这一节的问题是,在一个给定的图中,能找到多少条边不交的支撑树?如果不要求这些树不交,那么至少需要多少棵树才可以覆盖这个图呢?

这个问题直接应用于这样一个应用问题,在一个通信网络中,对于任意两个点,期望找到k条不同的路,倘若我们已经找出了这个图的k颗边不交的支撑树,那么显然这k个路可以直接被给出了。
\begin{thr}
一个多重图包含k颗边不交的支撑树,当且仅当对于任意顶点集的划分P,它有至少$|P|-1$条\textbf{交叉边}(端点位于不同部分的边)
\end{thr}
如果要有k颗不相交的支撑树,当然是需要有足够的连通性的,首先至少需要保证k-连通,但是这不是充分条件,下面的推论给出了连通程度与能否构造k颗不交支撑树的关系
\begin{coro}
2k-连通多重图有k颗边不交的支撑树。
\end{coro}
\end{document}