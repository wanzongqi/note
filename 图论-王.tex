\documentclass[UTF8]{ctexart}
\title{图论--王树禾}
\author{万宗祺}
\date{\today}
\usepackage{amsthm}
\usepackage{geometry}
\usepackage{amsfonts}
\usepackage{mathrsfs}
\usepackage{amsmath}
\geometry{left = 2cm,right=2cm,top=2.5cm,bottom=2.5cm}
\newtheorem{dfnt}{定义}
\newtheorem{thr}{定理}
\newtheorem{lemma}{引理}
\newtheorem*{coro}{推论}
\newtheorem*{note}{注}
\newtheorem{pro}{命题}
\newtheorem{alg}{算法}

\begin{document}
\maketitle
\tableofcontents
\section{图}
\subsection{图的基本概念}
\begin{dfnt}
数学结构$G=\{V(G),E(G),\Phi_G\}$为一个图,V是顶点集,E是边集,$\Phi_G$是E到$V \times V$的关联函数。
\end{dfnt}
\begin{dfnt}[图的一些基本定义]
\quad\\(1)边的端点\\(2)边与顶点相关联:边与他的两个端点是关联的\\(3)邻顶:同一个边的两个端点相邻\\(4)邻边:与同一个顶相关联的两条边叫做邻边\\(5)环:只与一个顶相关联的边叫做环\\(6)重边:端点一样的边是重边\\(7)完全图:任意两个顶点都相邻的图,记作$K_v$\\(8)单图:无环无重边的图\\(9)二分图:$V(G) = X \cup Y,X \cap Y = \emptyset$,且X中任意两个顶点不相邻,Y中任意两个顶点不相邻,则称G为二分图。若X中每个顶点与Y中每个顶点相邻,就称为完全二分图。记作$K_{m,n},m =|X|,n=|Y|$\\(10)星:$K_{1,n}$叫做星\\(11)完全r分图:同(9)类似\\(12)顶点的度数(次数):与该点相邻的边数,环计算两次,记作$d(v)$
\end{dfnt}
\begin{thr}[Euler]
$\sum \limits_{v \in V(G)} d(v) = 2\epsilon$
\end{thr}
\begin{coro}
图中奇次顶的总数是偶数
\end{coro}
\begin{dfnt}
同构,定义略
\end{dfnt}
\begin{dfnt}[边图]
设G是一个图,L(G)是另一个图,满足$V(L(G))=E(G),L(G)$中两顶相邻当且仅当他们是G中两条相邻的边,则称L(G)是G的边图。
\end{dfnt}
\begin{pro}
$G \cong L(G)$当且仅当G是多边形。
\end{pro}
\subsection{轨道和圈}
\begin{dfnt}
道路定义略,轨道就是各顶点不一样的道路,记作$P(v_0,v_k),v_0,v_k$是起点和终点,起点终点重合的道路叫做回路,起点终点重合的轨道叫做圈,长为k的圈叫做k阶圈,$u,v$两顶的距离指以u,v为起点的最短轨道长度,记作d(u,v)。若存在道路以$u,v$为起止点,则称u,v在G中连通,若G中任意两点都连通,则称G为连通图。
\end{dfnt}
\begin{dfnt}
子图概念略,若子图S顶点和图G顶点一样,就称子图为生成子图。若V(S)=V',E(S)是由两端都在V'中的边构成,就称S是由V'导出的G的导出子图。连通片定义略。
\end{dfnt}
\begin{pro}
|V|=2k,$d(v) \geq k$,则图连通
\end{pro}
\begin{pro}
在仅两个奇次顶的图中,这两个顶连通
\end{pro}
\begin{thr}
图G是二分图当且仅当G中无奇阶圈
\end{thr}
\begin{pro}
无零次与1次顶的单图中有圈
\end{pro}
\begin{note}
使用了“最长轨”技巧
\end{note}
\begin{dfnt}
单图G中最长的圈称为该图的周长;最短的圈之长称为该图的围长;两顶点间距离最大值称为直径,记作$d(G)$,图G的中心是指使$\max \limits_{v \in V(G)} d(u,v)$最小的顶u;这个最小值就是图的半径r(G)
\end{dfnt}
\begin{dfnt}
单图G的补图记作$G^C$,$V(G^C)=V(G)$,当且仅当在G中两顶不相邻,该二顶在$G^C$中相邻。
\end{dfnt}
\begin{pro}
单图G和他的补图不能都不连通
\end{pro}
\subsection{求最短轨长度的算法}
\begin{alg}[Dijkstra]
\quad \\(1)初始化边长度矩阵$\omega$,若两顶点不相邻,$\omega (uv) =\infty$\\(2)维护一个数组l,初始化l(v)=$\infty$,l($u_0$)=0;令$S_0$={$u_0$},i=0;\\(3)对每个不属于$S_i$的顶点v,赋值$l(v):=min\{l(v),l(u_i)+\omega (u_iv)\}$;设$u_{i+1}$是使$l(v)$取最小值的$V(G)-S_i$中的顶,令$S_{i+1} = S_i \cup \{u_{i+1}\}.$\\(4)i=v-1,就停止,否则用i+1代替i,转(3)
\end{alg}
\section{树}
\subsection{树的定义和性质}
\begin{dfnt}
无圈连通图称为树。每个连通片都为树的不连通图称为森林;树上次数为1的顶称为叶;如果一个树T是图G的生成子图,则称T是G的生成树,G-E(T)称为树余。
\end{dfnt}
\begin{thr}[树的等价命题]
若G是一个单图,则下面诸命题等价:\\(1)G是树\\(2)G中任二顶点间恰有一条轨\\(3)G中无圈,|E|=|V|-1\\(4)G是连通图,|E|=|V|-1\\(5)G是连通图,G-e不连通,e是G的任意一条边\\(6)G无圈,G+e恰含一个圈,其中e是任意一个不在E(G)中的以V(G)中的顶为端点的边
\end{thr}
\begin{coro}
G是连通图的充分必要条件是G有生成树
\end{coro}
\begin{pro}
$|V| \geq 2$的树T至少有两个叶。 
\end{pro}
\begin{pro}
连通图G的无圈子图是G的某个生成树的子图
\end{pro}
\subsection{生成树的个数}
\begin{dfnt}
用$\tau (G)表示G的生成树个数$
\end{dfnt}
\begin{thr}
$\tau (K_n)=n^{n-2}$ 
\end{thr}
\begin{thr}
e是连通图G中的一条边,则$$\tau (G) = \tau (G-e) + \tau (G \cdot e),$$其中$G \cdot e$是把边e的长度收缩成零,e的两个端点重合成一个顶形成的图。
\end{thr}
\begin{note}
用这个公式可以每次减少G的一条边,简化图来计算生成树个数
\end{note}
\subsection{求生成树的算法}
广度优先算法与深度优先算法,此处不再赘述
\subsection{求最优生成树算法}
\begin{alg}[Kruskal]
\quad \\(1)从E(G)中选一条权最小的边$e_1$\\(2)若$e_1,e_2,\cdots,e_i$已选出,则从E(G)-{$e_1,e_2,\cdots,e_i$}中选$e_{i+1}$,使得(i)G[{$e_1,e_2,\cdots,e_i$}]中无圈,(ii)$e_{i+1}$权值最小.\\(3)反复执行上述步骤直到选出$e_{v-1}$
\end{alg}
\subsection{有序二元树}
\begin{dfnt}
$d^+(v)$表示出度,$d^-(v)$表示入度。若T是树,把T的每一边标志一个方向,使得除顶$v_0 \in V(T)$外,每个$v \in V(T)-\{v_0\}$皆存在由$v_0$到v的有向轨,则称T是以$v_0$为根的外向树。
\end{dfnt}
\begin{dfnt}[Huffman树]
以$v_0$为根,$v_1,v_2,\cdots,v_n$为叶的有序二元树T中,$v_i$代表的事物出现的概率为$p_i$,满足$\sum \limits_{i=1}^{n} p_i = 1$,称轨$P(v_0,v_i)$的长为$v_i$的码长,且使得$$m(T)=\sum \limits_{i=1}^n p_i l_i = min$$则称T为带权$p_1,p_2,\cdots,p_n$的Huffman树,又称最优二元树。
\end{dfnt}
\begin{thr}
若T是Hoffman树,$p_1 \leq p_2 \leq \cdots \leq p_n,v_1,v_2,\cdots,v_n$为叶,(1)若$v_i$与$v_j$是兄弟,则$l_i=l_j$;(2)$v_1$与$v_2$是兄弟;(3)设$T^+$是带权$p_1+p_2,p_3,\cdots,p_n$的Hoffman树,与$p_1+p_2$相应的叶子生出两个新叶分别带权$p_1$和$p_2$,则得到带权$p_1,p_2,\cdots,p_n$的Hoffman树。
\end{thr}
\begin{note}
以上定理其实给出了生成Hoffman树的算法
\end{note}
\subsection{n顶有序编码二元树的数目}
\begin{note}
括号列技巧
\end{note}
\begin{thr}
n顶有序林与n顶有序编码二元树的数目皆为c(n).其中c(n)为Catalen数,$c(n)= \frac{1}{n+1}C_{2n}^{n}$
\end{thr}
\section{平面图}
\subsection{平面图及其平面嵌入}
\begin{dfnt}
把一个图G的图示画在曲面上,使得任二边不在内点相交,则称G可以嵌入这个曲面。若G可以嵌入平面,则称其为平面图。
\end{dfnt}
\begin{thr}
图G可平面嵌入的充分必要条件是G可以球面嵌入
\end{thr}
\begin{note}
球极平面投影,使球面顶点不在图的边或点上即可。
\end{note}
\subsection{Euler公式}
\begin{thr}[Euler公式]
G是连通平面图,则有公式$$v-\epsilon+\phi = 2$$其中v=V(G)是G的顶数,$\epsilon =\epsilon (G)$是G的边数,$\phi = \phi (G)$是G的面数
\end{thr}
\begin{dfnt}[面的度数]
记$F(G)=\{f_1,f_2,\cdots,f_{\phi}\}$是平面图G的面集合,$f_i$边界上的边条数记为$d(f_i)$,称为面的度数,是沿面边界行一周时,历经的边的条数,其中桥需要算两次
\end{dfnt}
\begin{pro}
$$\sum_{i=1}^{\phi}d(f_i)=2\epsilon$$
\end{pro}
\begin{note}
对面数量递推.Euler公式表明了,同样的图以不同方式嵌入不会改变面的数量。
\end{note}
\begin{coro}
若G是$v \geq 3$的连通平面图,则$\epsilon \leq 3v-6$.
\end{coro}
\begin{coro}
平面图G的最小顶次数$\delta \leq 5$
\end{coro}
\begin{note}
上面两个推论具体地告诉我们,图的边数如果过多就无法嵌入到平面中。
\end{note}
\subsection{极大平面图}
\begin{dfnt}[极大平面图]
若G是$v \geq 3$的平面图,当$u,v \in V(G),nv \notin E(G)$时,G+uv不再是平面图,则称G是极大平面图。
\end{dfnt}
\begin{thr}
$v \geq 3$的平面图G是极大平面图的充分必要条件是G的平面嵌入的每个面是三角形
\end{thr}
\begin{coro}
$v \geq 3$的平面图是极大平面图的充分必要条件是$\epsilon = 3v-6$
\end{coro}
\begin{thr}
G是$v \geq 4$的极大平面图,则$\delta (G) \geq 3$
\end{thr}
\subsection{平面图的充要条件}
\begin{thr}[Kuratowsky]
G是平面图当且仅当G中不含与$K_5$和$K_{3,3}$同胚的子图
\end{thr}
\begin{dfnt}
如果一个图不是平面图,那么我们可以把它的边分成n个部分,使每个部分的导出子图都可以镶嵌到一个平面中,最小的n就叫做这个图的厚度。
\end{dfnt}
\begin{thr}
若$\theta (G)$代表图G的厚度,则有以下估计\\(i)$\theta(G) \geq \{\frac{\epsilon}{3v-6}\},v>2$,{x}是x向上取整\\(ii)连通图G无三阶圈,则$\theta (G) \geq \{\frac{\epsilon}{2v-4}\},v>2$ \\(iii)$\theta (K_{v}) \geq [\frac{v+7}{6}],v\geq 3$
\end{thr}
\subsection{平面嵌入的灌木生长算法}
待补全。。。。
\section{匹配理论}
\subsection{匹配与许配}
\begin{dfnt}
\textbf{独立}边构成的集合M称为一个匹配,若U中的每个顶点都与M中一个边关联,就称M是U的一个匹配,或者说U中顶点被M匹配,而不与M中顶点相关联的就称为非匹配顶点,M中每边的两个端点称为在M中相配,M中每边的端点称为被M许配,若G中每个顶点都被M许配,称M为一个完备匹配,G中边数最多的匹配称作最大匹配。
\end{dfnt}
\begin{dfnt}[可增广轨]
设M是G中一个匹配,G中一条轨道P(u,v)上,u,v未被许配,但P(u,v)上的边交错出现在M中,就称P是一条可增广轨。由可增广轨可以把原匹配改进成一个多一条边的更大匹配(取对称差)
\end{dfnt}
\subsection{匹配定理}
\begin{thr}[Berge]
M是图G中的一个最大匹配当且仅当G中无M的可增广轨
\end{thr}
\begin{thr}[Hall]
设G是二分图,X,Y是它的两个部分。则存在把X中顶都许配的匹配的重要条件是$\forall S \subseteq X,|N(S)|\geq |S|$,其中$N(S)$是S中每个顶的邻顶组成的集合,称为邻集
\end{thr}
\begin{coro}
k次正则2分图有完备匹配
\end{coro}
\begin{dfnt}
设$B \subset V(G)$,G的每条边皆与B中顶关联,则称B是G的一个覆盖集;字面意思定义出极小覆盖,最小覆盖。最小覆盖的顶数目称为G的覆盖数,记为$\beta (G)$
\end{dfnt}
\begin{thr}[Konig]
若G是二分图,则其最大匹配的边数为$\beta (G)$
\end{thr}
\begin{thr}[Tutte]
图G有完备匹配当且仅当$\forall S \subset V(G),o(G-S) \leq |S|$,其中$o(G-S)$是G-S中奇数个顶点的连通片个数。
\end{thr}
\begin{thr}
无桥三次正则图有完备匹配
\end{thr}
\subsection{匹配算法}

\end{document}
