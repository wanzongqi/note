\documentclass[UTF8]{ctexart}
\title{图论--王树禾}
\author{万宗祺}
\date{\today}
\usepackage{amsthm}
\usepackage{geometry}
\usepackage{amsfonts}
\usepackage{mathrsfs}
\usepackage{amsmath}
\geometry{left = 2cm,right=2cm,top=2.5cm,bottom=2.5cm}
\newtheorem{dfnt}{定义}
\newtheorem{thr}{定理}
\newtheorem{lemma}{引理}
\newtheorem*{coro}{推论}
\newtheorem*{note}{注}
\newtheorem{pro}{命题}
\newtheorem{alg}{算法}

\begin{document}
\maketitle
\tableofcontents
\section{图}
\subsection{图的基本概念}
\begin{dfnt}
数学结构$G=\{V(G),E(G),\Phi_G\}$为一个图,V是顶点集,E是边集,$\Phi_G$是E到$V \times V$的关联函数。
\end{dfnt}
\begin{dfnt}[图的一些基本定义]
\quad\\(1)边的端点\\(2)边与顶点相关联:边与他的两个端点是关联的\\(3)邻顶:同一个边的两个端点相邻\\(4)邻边:与同一个顶相关联的两条边叫做邻边\\(5)环:只与一个顶相关联的边叫做环\\(6)重边:端点一样的边是重边\\(7)完全图:任意两个顶点都相邻的图,记作$K_v$\\(8)单图:无环无重边的图\\(9)二分图:$V(G) = X \cup Y,X \cap Y = \emptyset$,且X中任意两个顶点不相邻,Y中任意两个顶点不相邻,则称G为二分图。若X中每个顶点与Y中每个顶点相邻,就称为完全二分图。记作$K_{m,n},m =|X|,n=|Y|$\\(10)星:$K_{1,n}$叫做星\\(11)完全r分图:同(9)类似\\(12)顶点的度数(次数):与该点相邻的边数,环计算两次,记作$d(v)$
\end{dfnt}
\begin{thr}[Euler]
$\sum \limits_{v \in V(G)} d(v) = 2\epsilon$
\end{thr}
\begin{coro}
图中奇次顶的总数是偶数
\end{coro}
\begin{dfnt}
同构,定义略
\end{dfnt}
\begin{dfnt}[边图]
设G是一个图,L(G)是另一个图,满足$V(L(G))=E(G),L(G)$中两顶相邻当且仅当他们是G中两条相邻的边,则称L(G)是G的边图。
\end{dfnt}
\begin{pro}
$G \cong L(G)$当且仅当G是多边形。
\end{pro}
\subsection{轨道和圈}
\begin{dfnt}
道路定义略,轨道就是各顶点不一样的道路,记作$P(v_0,v_k),v_0,v_k$是起点和终点,起点终点重合的道路叫做回路,起点终点重合的轨道叫做圈,长为k的圈叫做k阶圈,$u,v$两顶的距离指以u,v为起点的最短轨道长度,记作d(u,v)。若存在道路以$u,v$为起止点,则称u,v在G中连通,若G中任意两点都连通,则称G为连通图。
\end{dfnt}
\begin{dfnt}
子图概念略,若子图S顶点和图G顶点一样,就称子图为生成子图。若V(S)=V',E(S)是由两端都在V'中的边构成,就称S是由V'导出的G的导出子图。连通片定义略。
\end{dfnt}
\begin{pro}
|V|=2k,$d(v) \geq k$,则图连通
\end{pro}
\begin{pro}
在仅两个奇次顶的图中,这两个顶连通
\end{pro}
\begin{thr}
图G是二分图当且仅当G中无奇阶圈
\end{thr}
\begin{pro}
无零次与1次顶的单图中有圈
\end{pro}
\begin{note}
使用了“最长轨”技巧
\end{note}
\begin{dfnt}
单图G中最长的圈称为该图的周长;最短的圈之长称为该图的围长;两顶点间距离最大值称为直径,记作$d(G)$,图G的中心是指使$\max \limits_{v \in V(G)} d(u,v)$最小的顶u;这个最小值就是图的半径r(G)
\end{dfnt}
\begin{dfnt}
单图G的补图记作$G^C$,$V(G^C)=V(G)$,当且仅当在G中两顶不相邻,该二顶在$G^C$中相邻。
\end{dfnt}
\begin{pro}
单图G和他的补图不能都不连通
\end{pro}
\subsection{求最短轨长度的算法}
\begin{alg}[Dijkstra]
\quad \\(1)初始化边长度矩阵$\omega$,若两顶点不相邻,$\omega (uv) =\infty$\\(2)维护一个数组l,初始化l(v)=$\infty$,l($u_0$)=0;令$S_0$={$u_0$},i=0;\\(3)对每个不属于$S_i$的顶点v,赋值$l(v):=min\{l(v),l(u_i)+\omega (u_iv)\}$;设$u_{i+1}$是使$l(v)$取最小值的$V(G)-S_i$中的顶,令$S_{i+1} = S_i \cup \{u_{i+1}\}.$\\(4)i=v-1,就停止,否则用i+1代替i,转(3)
\end{alg}
\section{树}
\subsection{树的定义和性质}
\begin{dfnt}
无圈连通图称为树。每个连通片都为树的不连通图称为森林;树上次数为1的顶称为叶;如果一个树T是图G的生成子图,则称T是G的生成树,G-E(T)称为树余。
\end{dfnt}
\begin{thr}[树的等价命题]
若G是一个单图,则下面诸命题等价:\\(1)G是树\\(2)G中任二顶点间恰有一条轨\\(3)G中无圈,|E|=|V|-1\\(4)G是连通图,|E|=|V|-1\\(5)G是连通图,G-e不连通,e是G的任意一条边\\(6)G无圈,G+e恰含一个圈,其中e是任意一个不在E(G)中的以V(G)中的顶为端点的边
\end{thr}
\begin{coro}
G是连通图的充分必要条件是G有生成树
\end{coro}
\begin{pro}
$|V| \geq 2$的树T至少有两个叶。 
\end{pro}
\begin{pro}
连通图G的无圈子图是G的某个生成树的子图
\end{pro}
\subsection{生成树的个数}
\begin{dfnt}
用$\tau (G)表示G的生成树个数$
\end{dfnt}
\begin{thr}
$\tau (K_n)=n^{n-2}$ 
\end{thr}
\begin{thr}
e是连通图G中的一条边,则$$\tau (G) = \tau (G-e) + \tau (G \cdot e),$$其中$G \cdot e$是把边e的长度收缩成零,e的两个端点重合成一个顶形成的图。
\end{thr}
\begin{note}
用这个公式可以每次减少G的一条边,简化图来计算生成树个数
\end{note}
\subsection{求生成树的算法}
广度优先算法与深度优先算法,此处不再赘述
\subsection{求最优生成树算法}
\begin{alg}[Kruskal]
\quad \\(1)从E(G)中选一条权最小的边$e_1$\\(2)若$e_1,e_2,\cdots,e_i$已选出,则从E(G)-{$e_1,e_2,\cdots,e_i$}中选$e_{i+1}$,使得(i)G[{$e_1,e_2,\cdots,e_i$}]中无圈,(ii)$e_{i+1}$权值最小.\\(3)反复执行上述步骤直到选出$e_{v-1}$
\end{alg}
\subsection{有序二元树}
\begin{dfnt}
$d^+(v)$表示出度,$d^-(v)$表示入度。若T是树,把T的每一边标志一个方向,使得除顶$v_0 \in V(T)$外,每个$v \in V(T)-\{v_0\}$皆存在由$v_0$到v的有向轨,则称T是以$v_0$为根的外向树。
\end{dfnt}
\begin{dfnt}[Huffman树]
以$v_0$为根,$v_1,v_2,\cdots,v_n$为叶的有序二元树T中,$v_i$代表的事物出现的概率为$p_i$,满足$\sum \limits_{i=1}^{n} p_i = 1$,称轨$P(v_0,v_i)$的长为$v_i$的码长,且使得$$m(T)=\sum \limits_{i=1}^n p_i l_i = min$$则称T为带权$p_1,p_2,\cdots,p_n$的Huffman树,又称最优二元树。
\end{dfnt}
\begin{thr}
若T是Hoffman树,$p_1 \leq p_2 \leq \cdots \leq p_n,v_1,v_2,\cdots,v_n$为叶,(1)若$v_i$与$v_j$是兄弟,则$l_i=l_j$;(2)$v_1$与$v_2$是兄弟;(3)设$T^+$是带权$p_1+p_2,p_3,\cdots,p_n$的Hoffman树,与$p_1+p_2$相应的叶子生出两个新叶分别带权$p_1$和$p_2$,则得到带权$p_1,p_2,\cdots,p_n$的Hoffman树。
\end{thr}
\begin{note}
以上定理其实给出了生成Hoffman树的算法
\end{note}
\subsection{n顶有序编码二元树的数目}
\begin{note}
括号列技巧
\end{note}
\begin{thr}
n顶有序林与n顶有序编码二元树的数目皆为c(n).其中c(n)为Catalen数,$c(n)= \frac{1}{n+1}C_{2n}^{n}$
\end{thr}
\section{平面图}
\subsection{平面图及其平面嵌入}
\begin{dfnt}
把一个图G的图示画在曲面上,使得任二边不在内点相交,则称G可以嵌入这个曲面。若G可以嵌入平面,则称其为平面图。
\end{dfnt}
\begin{thr}
图G可平面嵌入的充分必要条件是G可以球面嵌入
\end{thr}
\begin{note}
球极平面投影,使球面顶点不在图的边或点上即可。
\end{note}
\subsection{Euler公式}
\begin{thr}[Euler公式]
G是连通平面图,则有公式$$v-\epsilon+\phi = 2$$其中v=V(G)是G的顶数,$\epsilon =\epsilon (G)$是G的边数,$\phi = \phi (G)$是G的面数
\end{thr}
\begin{dfnt}[面的度数]
记$F(G)=\{f_1,f_2,\cdots,f_{\phi}\}$是平面图G的面集合,$f_i$边界上的边条数记为$d(f_i)$,称为面的度数,是沿面边界行一周时,历经的边的条数,其中桥需要算两次
\end{dfnt}
\begin{pro}
$$\sum_{i=1}^{\phi}d(f_i)=2\epsilon$$
\end{pro}
\begin{note}
对面数量递推.Euler公式表明了,同样的图以不同方式嵌入不会改变面的数量。
\end{note}
\begin{coro}
若G是$v \geq 3$的连通平面图,则$\epsilon \leq 3v-6$.
\end{coro}
\begin{coro}
平面图G的最小顶次数$\delta \leq 5$
\end{coro}
\begin{note}
上面两个推论具体地告诉我们,图的边数如果过多就无法嵌入到平面中。
\end{note}
\subsection{极大平面图}
\begin{dfnt}[极大平面图]
若G是$v \geq 3$的平面图,当$u,v \in V(G),nv \notin E(G)$时,G+uv不再是平面图,则称G是极大平面图。
\end{dfnt}
\begin{thr}
$v \geq 3$的平面图G是极大平面图的充分必要条件是G的平面嵌入的每个面是三角形
\end{thr}
\begin{coro}
$v \geq 3$的平面图是极大平面图的充分必要条件是$\epsilon = 3v-6$
\end{coro}
\begin{thr}
G是$v \geq 4$的极大平面图,则$\delta (G) \geq 3$
\end{thr}
\subsection{平面图的充要条件}
\begin{thr}[Kuratowsky]
G是平面图当且仅当G中不含与$K_5$和$K_{3,3}$同胚的子图
\end{thr}
\begin{dfnt}
如果一个图不是平面图,那么我们可以把它的边分成n个部分,使每个部分的导出子图都可以镶嵌到一个平面中,最小的n就叫做这个图的厚度。
\end{dfnt}
\begin{thr}
若$\theta (G)$代表图G的厚度,则有以下估计\\(i)$\theta(G) \geq \{\frac{\epsilon}{3v-6}\},v>2$,{x}是x向上取整\\(ii)连通图G无三阶圈,则$\theta (G) \geq \{\frac{\epsilon}{2v-4}\},v>2$ \\(iii)$\theta (K_{v}) \geq [\frac{v+7}{6}],v\geq 3$
\end{thr}
\subsection{平面嵌入的灌木生长算法}
待补全。。。。
\section{匹配理论}
\subsection{匹配与许配}
\begin{dfnt}
\textbf{独立}边构成的集合M称为一个匹配,若U中的每个顶点都与M中一个边关联,就称M是U的一个匹配,或者说U中顶点被M匹配,而不与M中顶点相关联的就称为非匹配顶点,M中每边的两个端点称为在M中相配,M中每边的端点称为被M许配,若G中每个顶点都被M许配,称M为一个完备匹配,G中边数最多的匹配称作最大匹配。
\end{dfnt}
\begin{dfnt}[可增广轨]
设M是G中一个匹配,G中一条轨道P(u,v)上,u,v未被许配,但P(u,v)上的边交错出现在M中,就称P是一条可增广轨。由可增广轨可以把原匹配改进成一个多一条边的更大匹配(取对称差)
\end{dfnt}
\subsection{匹配定理}
\begin{thr}[Berge]
M是图G中的一个最大匹配当且仅当G中无M的可增广轨
\end{thr}
\begin{thr}[Hall]
设G是二分图,X,Y是它的两个部分。则存在把X中顶都许配的匹配的重要条件是$\forall S \subseteq X,|N(S)|\geq |S|$,其中$N(S)$是S中每个顶的邻顶组成的集合,称为邻集
\end{thr}
\begin{coro}
k次正则2分图有完备匹配
\end{coro}
\begin{dfnt}
设$B \subset V(G)$,G的每条边皆与B中顶关联,则称B是G的一个覆盖集;字面意思定义出极小覆盖,最小覆盖。最小覆盖的顶数目称为G的覆盖数,记为$\beta (G)$
\end{dfnt}
\begin{thr}[Konig]
若G是二分图,则其最大匹配的边数为$\beta (G)$
\end{thr}
\begin{thr}[Tutte]
图G有完备匹配当且仅当$\forall S \subset V(G),o(G-S) \leq |S|$,其中$o(G-S)$是G-S中奇数个顶点的连通片个数。
\end{thr}
\begin{thr}
无桥三次正则图有完备匹配
\end{thr}
\subsection{匹配算法}
\begin{alg}[匈牙利算法]
\item (1)设G是连通二分图,在G中任取初始匹配M
\item (2)若M把X中顶全部匹配,止,M就是最大匹配;否则取X中未被匹配的顶u,令S=\{u\},T=$\emptyset$
\item (3)若N(S)=T,中止,否则取$y \in N(S)-T$
\item (4)若y被M许配,设$yz \in M$,$S \leftarrow S\cup\{z\},T \leftarrow T \cup \{y\}$,转(3);否则P(u,y)是增广轨,与原匹配取对称差可得到更大的匹配,然后转(2)
\end{alg}
\begin{note}
具体实现算法的时候需要记录增广轨的轨迹。Berge定理保证了算法的正确性。
\end{note}
接下来的问题是最佳匹配问题,问题模型为,对完全二分图$G = K_{n,n}$的每个边e,加权$\omega (e) \geq 0$,求总权最大的匹配。
\begin{dfnt}
对$K_{n,n}$每顶给一个顶标$l(v)$,当$x \in X,t \in Y,l(x)+l(y) \geq \omega(xy)$时,称此种顶标为正常顶标。正常顶标是存在的$$l(x)=\max \limits_{y \in Y} \omega(xy) ,x \in X$$
$$l(y) = 0, y\in Y$$就是一个正常顶标。
\end{dfnt}
\begin{thr}
设$G=K_{n,n}$是具有正常顶标l的加权图,取G的边子集$$E_l = \{xy | xy \in E(G),l(x)+l(y)=\omega (xy)\}$$令$G_l$是以$E_l$为边集的生成子图,如果$G_l$有完备匹配M,则M是G的最佳匹配
\end{thr}
而如果没有完备匹配,KM算法给出了一种不断修改顶标使得最终可以达到完备匹配的方法
\begin{alg}[KM算法]
\item (1)设G是连通二分图,在G中任取初始匹配M和初始顶标
\item (2)若M把X中顶全部匹配,止,M就是最大匹配;否则取X中未被匹配的顶u,令S=\{u\},T=$\emptyset$
\item (3)若N(S)=T,且X未被全部匹配,取$$a_l = \min \limits_{x \in S ,y\notin T}\{l(x)+l(y)-\omega(xy)\}$$ 
\[\bar{l} =\left \{ \begin{array}{ll} l(v)-a_l & v \in S \\l(v)+a_l &v \in T \\l(v) &其它 \end{array}\right.\]
$l \leftarrow \bar{l},G_l \leftarrow G_{\bar{l}}$
否则取$y \in N(S)-T$
\item (4)若y被M许配,设$yz \in M$,$S \leftarrow S\cup\{z\},T \leftarrow T \cup \{y\}$,转(3);否则P(u,y)是增广轨,与原匹配取对称差可得到更大的匹配,然后转(2)
\end{alg}
\begin{note}
相比起匈牙利算法,多了一步修改顶标操作
\end{note}
\subsection{图的因子分解}
\begin{dfnt}
将给定的图G分解成若干两两无公共边的生成子图$G_1,G_2\cdots G_n$,且要求$G_i$具有某种性质,即$V(G_i)=V(G),\bigcup \limits_{i=1}^{n}E(G_i)=E(G),E(G_i) \cap E(G_j) = \emptyset$,并且要求$G_i$有性质$p_i$。称$G_i$是G的因子,这个过程叫因子分解。一个因子是m次正则图时,称此因子是G的m因子;如G的因子全部是m因子,则称G可以m因子分解。
\end{dfnt}
\begin{thr}
$K_{2n}$可以1因子分解
\end{thr}
\begin{thr}
$K_{2n+1}$存在每个因子皆生成圈的2因子分解,共计n个生成圈
\end{thr}

\section{着色理论}
\subsection{边着色}
略
\subsection{顶着色}
\begin{dfnt}
如果使用n种颜色把图G的每顶分配一种颜色,使得邻顶异色,则称此为对G的顶的正常着色。图G的顶的正常着色种所需颜色数最小值为G的顶色数,简称色数,记为$\xi$,色数为k的图称为k色图
\end{dfnt}
\begin{pro}
\item (1)G是有边二分图的充要条件是$\xi (G)=2$
\item (2)G是无边图当且仅当$\xi (G)=1$
\item (3)G是完全图当且仅当$\xi (G)=|V(G)|$
\item (4)$\xi (G) \leq \Delta (G)+1$
\end{pro}
\begin{thr}
平面图色数不大于4
\end{thr}
\subsection{颜色多项式}
\begin{dfnt}
用$P(G,k)$表示对图G用k种颜色顶正常着色的方式数,若G给定,则其是关于k的一元函数。但G不同,这个函数$P(G,k)$未必相同。
\end{dfnt}
\begin{pro}
\item (1)$\xi (G)\leq k$的充要条件是P(G,k)>0
\item (2)$P(G,k)=k(k-1)\cdots (k-v+1)$的充要条件是$G = K_v$
\item (3)$P(G,k)=k^v$的充要条件是$|E(G)|=0$
\item (4)若$G_1,G_2,\cdots,G_{\omega}$是G的连通片,则$P(G,k)=\prod \limits_{i=1}^{\omega}P(G_i,k)$
\item (5)$P(G-e,k)=P(G,k)+P(G\cdot e,k)$
\end{pro}
\begin{thr}
$P(G,k)$是k的v次多项式,$v=|V(G)|$,且按降幂排列,首项为$k^v$,第二项为$-\epsilon k^{v-1}$,无常数项,系数为正负交错的整数。
\end{thr}
\begin{thr}
图G的颜色多项式为$k(k-1)^{v-1}$当且仅当G是v个顶的树
\end{thr}
\subsection{独立集}
\begin{dfnt}
设$I \subseteq V(G),\forall u,v \in I,$u与v不相邻,就说I是一个独立集。按通常定义方式定义出极大独立集和最大独立集。
\end{dfnt}
\begin{dfnt}
若把图G的顶集V(G)划分成$V_1,V_2,\cdots,V_k$这k个子集,即$V(G)=\bigcup \limits_{i=1}^k V_i,V_i \cap V_j=\emptyset,i \neq j,V_i$是$G-\bigcup \limits_{j=1}^{i-1}V_j$的极大独立集,其中$V_0 = \emptyset$,把$V_i$种的顶染上i色,则称这种上色是对G的一种k顶规范着色。
\end{dfnt}
\begin{note}
规范着色一定是正常着色,实际上,规范着色是在正常着色种按顺序尽量多着色一些无色边。
\end{note}
\begin{thr}
如果G可以k顶正常着色,则G存在k顶规范着色。
\end{thr}
\begin{pro}[独立集与覆盖集]
\item (1)I为G的独立集的充要条件是V(G)-I是G的覆盖集
\item (2)I为G的极大独立集,则$V(G)-I$是G的极小覆盖集
\item (3)$\alpha (G)+\beta (G) = |V(G)|$
\end{pro}
\begin{alg}[求极小覆盖集]
规定:
\item a或b,写成a+b
\item a与b,写成ab
$\forall v \in V(G)$,为了覆盖一切与v关联的边,或是v参加覆盖集,或是v的一切邻顶参加覆盖集,可以写成$$v+\prod \limits_{u \in N(v)}u$$所以极小覆盖集由下面多项式给出$$\prod_{v \in V(G)}(v+\prod \limits_{u \in N(v)}u)$$此式每一项是一个极小覆盖集
\end{alg}
\begin{dfnt}
$V_1 \subseteq V(G),\forall v \in V(G)$,则$v \in V_1$,不然v与$V_1$内一顶相邻,则称$V_1$为G的最小支配集。同字面意思定义极小与最小支配集。
\end{dfnt}
\begin{note}
任何图的极大独立集必为极小支配集

按逻辑运算展开下式,每一项给出一个极小支配集$$\prod_{v \in V(G)}(v+\sum \limits_{u \in N(v)}u)$$事实上,在G中,为了v接收支配,或者v参与支配集,或者v的某个邻顶参与。
\end{note}

\subsection{Ramsey数}
\section{Euler图与Hamilton图}
\subsection{Euler图}
\begin{dfnt}
在图G中含一切边的行迹叫做Euler行迹,含一切边的闭行迹叫做Euler回路,若G中存在Euler回路,则称G为Euler图。
\end{dfnt}
\begin{thr}
对于连通图G,(1)G是Euler图的充分必要条件是$\forall v\in V(G),d(v)$是偶数,(2)G是Euler图的充要条件是G可表为无公共边的圈之并。
\end{thr}
\begin{thr}
连通图G是Euler行迹当且仅当G中至多两个奇次项。
\end{thr}
\subsection{中国邮递员问题}
\begin{dfnt}[问题模型]
任给一个图G,对E(G)加权,即对每个$e \in E(G)$,任意指定一个非负实数$\omega(e)=min$
\end{dfnt}
\begin{alg}[求Euler回路的FE算法]
\item (1)任取$v_0 \in V(G),令W_0=v_0$
\item (2)设行迹$W_i=v_0v_1\cdots v_i$已选定,则从E(G)-E(W)中选一条边$e_{i+1}$使得$e_{i+1}$与$v_i$关联,且非必要时,$e_{i+1}$不要选$G-E(W)$的桥.
\item (3)反复执行(2)
\end{alg}
\begin{alg}[中国邮递员算法]
\item (1)求G中奇次顶集合$V_0$
\item (2)对$V_0$中每个顶对,用dijsktra算法求距离d(u,v)
\item (3)构作加权完全图$K_{|V_0|}$,$V(K_{|V_0|})=V_0,K_{|V_0|}$中u,v权为d(u,v).
\item (4)求加权图$K_{|V_0|}$的总权最小的完备匹配
\item (5)在G中求M中同一边之端点间的最短轨
\item (6)把G中在5求得的每条最短轨之边变成同权倍边,得到Euler图G'(这个图没有奇顶点)
\item (7)用FE算法求G‘一条Euler回路,即G的中国邮路
\end{alg}
\subsection{Hamilton}
\begin{dfnt}
称含图的一切顶的圈为Hamilton圈,有Hamilton圈的图为Hamilton图;把含图的一切顶的轨称为Hamilton轨。
\end{dfnt}
\begin{thr}
G是Hamilton图的必要条件是任取$S \subset V(G),S\neq \empty$,则$\omega(G-S) \leq |S|$,其中$\omega (\cdot)$是连通片个数.
\end{thr}
\begin{thr}
设$|V(G)| \geq 3$,G的任一对顶u,v皆有$d(u)+d(v) \geq |V(G)|-1$,则G中有Hamilton轨;若$d(u) +d(v) \geq |V(G)|$,则G是Hamilton图
\end{thr}
\section{有向图}
\subsection{弱连通,单连通与强连通}
\begin{dfnt}
若G是有向图,如果对$u,v \in V(G)$,存在从u到v的有向轨,则称u可达v;$\forall u,v \in V(G)$,u可达v或v可达u时,则称G是单连通有向图;$\forall u,v \in G$,不但u可达v,而且v可达u,则称G是强连通有向图;若把G的每边箭头取消,得到的无向图称为G的底图,底图连通的有向图叫做弱连通有向图。
\end{dfnt}
\begin{thr}
G是强连通有向图当且仅当G中存在有向生成回路(包含所有点的有向回路)
\end{thr}
\begin{thr}
当且仅当无向图是无桥连通图时,G可以定向成强连通有向图
\end{thr}
\begin{thr}
G为单连通有向图当且仅当G中有含G所有点的有向道路
\end{thr}
\begin{pro}
G是弱连通有向图时,G为有向Euler图的充要条件是对每个顶$v \in V(G)$,皆成立$d^-{v}=d^+{v}$,其中$d^-{v},d^+{v}$是v的入度和出度
\end{pro}
\begin{pro}
G是弱连通有向图,G为有向Euler图的充要条件是$G = \bigcup \limits_{i=1}^n C_i$,$C_i$是G中有向圈,且$E(C_i)\cap E(C_j)=\emptyset , 1\leq i,j \leq n$,n是某个自然数
\end{pro}
\begin{pro}
若G是弱连通有向图,且
\[d^-(v)=\left \{ \begin{array}{ll} d^+(v) & v\in V(G) - \{u_1,u_2\},u_1,u_2 \in V(G);\\d^+(v) -1 &v = u_1;\\d^-(v) +1 &v = u_2 \end{array}\right.\]
则G中存在从$u_1$到$u_2$的有向Euler行迹
\end{pro}
\subsection{循环赛图,有向轨和王}
所谓循环赛图,又称竞赛图或赛图,是一个无向完全图$K_n$,把其每个边加一个方向而得到的有向图。它的实际背景是n位运动员,每位选手都要和其它选手比赛一场,当甲选手胜乙选手时,加一条甲到乙的边。
\begin{thr}
若有向图底图为G,则此有向图有长$\xi (G)-1$的有向轨
\end{thr}
\begin{note}
由于$\xi (K_v) =v$,故每个赛图都有Hamilton轨
\end{note}
\begin{dfnt}
在赛图中,若从一个顶出发,通过至多长2的顶可达任何一个顶,就称这个顶为王。顶的得分为顶胜利的次数。
\end{dfnt}
\begin{thr}
赛图中得分最多的顶为王
\end{thr}
\begin{thr}
赛图G中v为唯一王的充要条件是u得分为1,其中n是赛图顶数
\end{thr}
\subsection{有向Hamilton图}
\begin{dfnt}
$$\delta^- = \min_{v \in V(G)}\{d^-(v)\}$$
$$\delta^+ = \min_{v \in V(G)}\{d^+(v)\}$$
\end{dfnt}
\begin{thr}[泛圈定理]
设G是强连通竞赛图,$\forall u \in V(G)$,则G中存在k阶圈,u在此k阶圈上,k=$3,4,\cdots,|V(G)|$
\end{thr}
\begin{thr}
设$P(u_0,v_0)$$|E(P(u_0,v_0))| \geq \max \{\delta^-,\delta^+\}$
\end{thr}
\begin{coro}
严格有向图中有长度大于$\max\{\delta^+,\delta^-\}$的有向圈
\end{coro}
\begin{thr}
$\min\{\delta^-,\delta^+\} \geq \frac{1}{2} |V(G)| >1$的严格有向图G是有向Hamilton图.
\end{thr}
\section{最大流算法}
\subsection{2F算法}
最大流问题的实际问题原型如下:

把一种商品从产地通过铁路或公路运往市场,交通网络中每一路段的运输能力有一定限度.问如何安排运输,使得运输最快?

图论模型如下:

设G是弱连通严格有向加权图,$s \in V(G)$称为源,$t \in V(G) -\{s\}$称为汇,每边e之权c(e)称为边容量,这时称G上定义了一个网络,记成$N(G,s,t,c(e))$.在E(G)上定义一个函数$f(e):E \rightarrow \mathbb{R},f(e)$满足\[C(1) \quad 0\leq f(e)\leq c(e), \quad e \in E(G)\] \[C(2) \quad \sum_{e \in \alpha(v)} f(e) = \sum_{e\in \beta(v)} f(e),\quad v \in V(G)-\{s,t\}\]其中$\alpha(v)$是以v为头的边集,$\beta(v)$是以v为尾的边集,称上述函数为网络N上的流函数,简称流,称$$F=\sum_{e\in \alpha(t)}f(e)-\sum_{e\in \beta(t)}f(e)$$为流函数流量.

目标是使F最大
\begin{dfnt}
设$S \subset V(G),s \in S,t \in \bar{S}=V(G)-S$.则称$(S,\bar{S})$是网络的一个截.\[C(S) = \sum_{e\in (S,\bar{S})}c(e)\]
\end{dfnt}
\begin{note}
截量表示从S中的各站运往S以外各站的货物量的一个上界.由于除s与t外,其他站皆中转站,所以C(S)也就是单位时间从s运往t的货物量的一个上界.
\end{note}
\begin{thr}
对于任一截$(S,\bar{S})$,成立公\[F = \sum_{e \in (S,\bar{S})}f(e) - \sum_{e\in (\bar{S},S)} f(e)\]
\end{thr}
\begin{coro}
对任何流函数f和任意截($S,\bar{S}$)\[F \leq C(S)\]
\end{coro}
\begin{coro}
若F=C(S),则F是最大流量,C(S)是最小截量.
\end{coro}
\begin{dfnt}[2F算法中的一些规定]
\item (1).若$e=uv \in E(G)$,u已有标志,v尚未标志,且边e未满载,即c(e)>f(e),则称沿e可前向标志v,且规定$$\Delta (e) = c(e)-f(e)$$标志了边e.
\item (2).若$e=xy \in E(G)$,y已有标志,而x尚未标志,且f(e)>0,则称沿e可向后标志顶x,且规定$$\Delta (e) = f(e)$$标志了边e.
\end{dfnt}
\begin{alg}[2F算法]
\item (1)取初始流函数$f(e) = 0,\forall e \in E(G)$
\item (2)标志顶s,其它顶未标志
\item (3)选一个可向前或可向后标志的顶v,若选不到这种顶就中止,得到的就是最大流,若可选到v,就标志v且标志边e;若v=t,转(4),否则转(3).
\item (4)设已得到一条标志了的无向轨$se_1v_1e_2v_2\cdots e_lt$,取$\Delta = \min \limits_{1 \leq i\leq l}\{\Delta (e_i)\}$,若在有向图G中$e_i=u_{i-1}v_i(s=v_0,t=v_l)$则\[f(e_i)\leftarrow f(e_i)+\Delta \]若$e_i$在有向图G中为$e_i = v_iv_{i-1}$则\[f(e_i)\leftarrow f(e_i)-\Delta\]
\item (5)转(2)
\end{alg}

\begin{thr}
2F算法终止时的流函数时最大流,其流量等于G中最小截量
\end{thr}
\begin{coro}[双最定理]
f是最大流,$(S,\bar{S})$是最小截的充要条件是f的流量F等于$(S,\bar{S})$的截量C(S).
\end{coro}
\subsection{Dinic分层算法}
2F算法在面对图中存在小权边时,每次增加的流量被小权边限制,从而要花费大量迭代才能求得最大流,Dinic算法则避免了这个缺点

Dinic把2F的“沿e可向前标志”或“沿e可向后标志”的边e称为“有用边”
\begin{alg}[Dinic分层算法]
\item (1)$V_0 \leftarrow \{s\},i\leftarrow 0$
\item (2)$T\leftarrow \{v|v \notin V_j,j\leq i,\text{且存在从$V_i$中某顶与v之间的有用边} \}$
\item (3)若$T = \emptyset$,止,现在的流就是最大流
\item (4)若$t\in T$,则$l\leftarrow i+1,V_l \leftarrow \{t\}$,止
\item (5)令$V_{i+1}\leftarrow T$增大i,转(2)

上述算法得到的$V_i$叫做第i层,仅相邻两层间有边,得到的网络叫层状网络.
\end{alg}
\begin{thr}
若Dinic终止于(3),则现网络上的流f是最大流.
\end{thr}
若在(4)终止,就需要在已经分层的网络上求得最大流

设原来网络中的流为f,$E_j$是层状网络中从$V_{j-1}$到$V_j$的边子集,若$e=uv \in E_j$
\[
\bar{c}(e) = \left \{
\begin{array}{lll}
c(e)-f(e) &u\in V_{j-1}&v\in V_{j}\\
f(e) &v \in V_{j-1}&u\in V_{j}
\end{array}
\right.\]
$\bar{c}$是$e=uv$上可增载的上界,我们把层状网络的边权取成$\bar{c}e$,取初始流$\bar{f}(e)=0$,且把层状网络上的边定向成从$V_{j-1}$到$V_j$(边的原来方向可能变成其反向)

若在层状网络中,每条从s到t的轨$$sv_1v_2\cdots v_{l-1}t$$上至少有一边$e_j$,满足$\bar{f}(e_j) = \bar{c}(e_j)$,其中$v_j \in V_j,e_j \in E_j$,则称$\bar{f}$为层状网络上的一个极大流.

层状网络上的极大流未必是最大流,但如果能求得层状网络的一个极大流,则可把原来网络上的流f改进成更大的流$f'$:
\[
f'(e)=\left \{
\begin{array}{ll}
f(e)+\bar{f}(e)&e=uv\in E(G),u\in V_{j-1},v\in V_j\\
f(e)-\bar{f}(e)&e=vu\in E(G),u\in V_{j-1},v\in V_j\\
f(e)&otherwise
\end{array}
\right.\]

理论证明了每一轮分层层数严格增加,那么在有限轮后,算法必定在(3)终止,求得了最大流,下面只需要给出在层状网络求极大流的算法
\begin{alg}[极大流算法]
\item (1)把层状网络$\bar{N}$上每条边标志“未堵塞”,$\bar{f}\leftarrow 0$
\item (2)$v\leftarrow s,S = \emptyset$
\item (3)若无未堵塞的边$e=vu,$u在下一层,则执行
\item (3.1)若s=v,止,$\bar{f}$即极大流
\item (3.2)从S移除顶部的边e=uv
\item (3.3)标志e堵塞,$v\leftarrow u$
\item (3.4)转(3).
\item (4)选一未堵塞的边e=vu,u在下一层,把e放入S,$v\leftarrow u$,若$v\neq t$,转(3)
\item (5)S中的边构成一个可增载轨$$se_1v_1e_2\cdots v_{l-1}e_lt$$
\item (5.1)$\Delta \leftarrow \min \limits{1\leq i \leq l}\{\bar{c}(e_i)-\bar{f}(e_i)\}$
\item (5.2)对每个$1\leq i \leq l,\bar{f}(e_i)\leftarrow \bar{f}(e_i)+\Delta$,当$\bar{f}(e_i)=\bar{c}(e_i)$,标志$e_i$堵塞
\item (5.3)转(2)
\end{alg}
\begin{note}
具体实现应该要利用深度优先搜索,思想还是寻找增广轨
\end{note}
\subsection{有上下界网络的最大流算法}
现在对于上一节中的问题模型,有向图G每边有了两个权$b(e),c(e)$,同时C1条件变为\[b(e)\leq f(e) \leq c(e)\]

对这种网络,可能不存在流函数,需要先解决存在判定问题.

\begin{dfnt}[伴随网络]
N(G,s,t,b(e),c(e))的伴随网络N'(G'(V',E'),s',t',b'(e),c'(e)):
\item (i)V' = $\{s',t'\}\cup V(G)$,其中$s',t' \notin V(G)$
\item (ii)$\forall v \in V(G)$,加新边$e=vt'$,且令$c'(e)=\sum \limits_{e \in \beta(v)}b(e),c'(e)$是N'中e的容量上界,下界$b'(e)=0$
\item (iii)$\forall v \in V(G)$,加新边$e=s'v$,且令$c'(e)=\sum \limits_{e \in \alpha(v)}b(e),c'(e)$是N'中e的上界,下界b'(e)=0
\item (iv)E(G)中的边e在N'中仍保留,但权发生变化,下界变成0,上界$c'(e)=c'(e')=+\infty$
\item (v)加新边e=st与e'=ts,且令e与e'的下界b'(e)=b'(e')=0,上界$c'(e)=c'(e)=+\infty$
\end{dfnt}
\begin{thr}
N(G,s,t,b(e),c(e))存在可行流当且仅当伴随网络N'上的最大流f'使流出s'的一切边e皆满足f'(e)=c'(e).这时f'(e)+b(e)是N上一个可行流.
\end{thr}
\begin{note}
由以上,得到求上下界网络N的最大流步骤:
\item (1)求出N的伴随网络N'
\item (2)求N'的最大流f'
\item (3)检验f'是否使N'中出s'的边e皆满足f'(e)=c'(e),否则,无可行流,是,有可行流f(e)$$f(e)=f'(e)+b(e)$$
\item (4)若已经得到可行流f(e),则基于f(e)用2F或Dinic求最大流(这时可以不考虑下界)
\end{note}
\section{连通度}
\subsection{顶连通度}

\end{document}
